\documentclass[10pt,a4paper]{report}
\usepackage{bsymb,b2latex}
\begin{document}
\SingleHeader{110-MappingContext}
\CONTEXT{110-MappingContext}{040-OntologyContext\_Particulars}{}
\SETS{
	\Set{E}{l'ensemble des expressions de la source de données}
	\Set{K}{l'ensemble des clés des expressions (composants de signature, une clé peut être composée de plusieurs composants)}
}
\CONSTANTS{
	\Constant{individuation}{les particuliers individualisés par l'expression}
	\Constant{exp\_subsets}{les particuliers individualisés par une expression sont aussi individualisés par ces expressions (e1 $\in{}$ exp$\_$subsets[$\{$e2$\}$]: e1 est un sous-ensemble de e2)}
	\Constant{exp\_equivalent}{tous les particuliers individualisés par l'expression sont aussi individualisés par ces expressions}
	\Constant{exp\_overlap}{certains particuliers individualisés par une expression sont individualisés par d'autres expressions}
	\Constant{exp\_key}{relation entre une expression et sa clé}
	\Constant{key\_compatible}{la relation entre deux clés compatibles}
	\Constant{exp\_list}{la liste de toutes les expressions, dans un ordre quelconque (utilisé pour certaines preuves, utile pour relever une assignation arbitraire)}
}
\AXIOMS{
	\Axiom{map\_finite\_E}{false}{$finite(E)$}{\\E est un ensemble fini}
	\Axiom{map\_finite\_K}{false}{$finite(K)$}{\\K est un ensemble fini}
	\Axiom{map\_type\_individuation}{false}{$individuation \in{} E \srel{} P$}{\\relation surjective des expressions vers les particuliers dont elles permettent l'individuation}
	\Axiom{map\_type\_expSubsets}{false}{$exp\_subsets \in{} E \rel{} E$}{\\relation entre une expression e1 et une expression e2, avec les particuliers individualisés par e1 qui sont également individualisés par e2  }
	\Axiom{map\_type\_expEquivalent}{false}{$exp\_equivalent \in{} E \rel{} E$}{\\relation entre une première expression et une seconde expression qui individualise tous les particuliers individualisés par la première }
	\Axiom{map\_type\_expOverlap}{false}{$exp\_overlap \in{} E \rel{} E$}{\\relation entre des expressions qui se chevauchent, c'est-à-dire qu'il y a intersection de leurs particuliers}
	\Axiom{map\_type\_expKey}{false}{$exp\_key \in{} E \tsur{} K$}{\\fonction surjective entre une expression et sa clé}
	\Axiom{map\_type\_compatibleKey}{false}{$key\_compatible \in{} K \rel{} K$}{\\relation entre plusieurs clés qui sont notées comme compatibles}
	\Axiom{map\_type\_expList}{false}{$exp\_list \in{} 1\upto{}card(E) \tbij{} E$}{\\fonction bijective entre la numérotation de la liste et les éléments de E}
	\Axiom{map\_individuation\_def}{false}{$\btrue{}$}{\\(axiome retiré pour l'animation) tout sous-ensemble de P est représenté par au moins une expression}
	\Axiom{map\_expSubsets\_def}{false}{$\\\forall{}e1,e2 \qdot{} \{e1,e2\} \subseteq{} E \land{}\\~        e1 \in{} exp\_subsets[\{e2\}] \\~     \leqv{} \\~        individuation[\{e1\}] \subseteq{} individuation[\{e2\}]$}{\\une expression e1 est sous-ensemble d'une expression e2 si l'ensemble de particuliers de e1 est un sous-ensemble de l'ensemble de particuliers de e2}
	\Axiom{map\_expSubsets\_thm\_reflexive}{true}{$\forall{}e \qdot{} e \in{} E \limp{} e \in{} exp\_subsets[\{e\}]$}{\\exp$\_$subsets est une relation réflexive}
	\Axiom{map\_expSubsets\_thm\_transitive}{true}{$\\\forall{}e1,e2,e3 \qdot{} e1 \in{} exp\_subsets[\{e2\}] \land{} \\~           e2 \in{} exp\_subsets[\{e3\}]\\~        \limp{} e1 \in{} exp\_subsets[\{e3\}]$}{\\exp$\_$subsets est une relation transitive}
	\Axiom{map\_expOverlap\_def}{false}{$\\\forall{}e1,e2 \qdot{} \{e1,e2\} \subseteq{} E \land{}\\~        e1 \in{} exp\_overlap[\{e2\}] \\~     \leqv{} (individuation[\{e1\}] \binter{} individuation[\{e2\}]) \neq{} \emptyset{} \lor{} \\~         individuation[\{e1\}] = \emptyset{} \lor{} \\~         individuation[\{e2\}] = \emptyset{}$}{\\une expression e1 overlap une expression e2 si l'intersection des particuliers qu'ils individualisent n'est pas vide ou si une des deux expressions n'individualise aucun particulier}
	\Axiom{map\_expOverlap\_thm\_nonEmpty}{true}{$\\\forall{}e1,e2 \qdot{} \{e1,e2\} \subseteq{} E \land{}\\~        e1 \in{} exp\_overlap[\{e2\}] \land{}\\~        individuation[\{e1\}] \neq{} \emptyset{} \land{}\\~        individuation[\{e2\}] \neq{} \emptyset{}\\~     \leqv{} (individuation[\{e1\}] \binter{} individuation[\{e2\}]) \neq{} \emptyset{}$}{\\deux expressions non vide s'overlap si l'intersection des particuliers qu'ils individualisent n'est pas vide}
	\Axiom{map\_expOverlap\_thm\_reflevive}{true}{$\forall{}e \qdot{} e \in{} E \limp{} e \in{} exp\_overlap[\{e\}]$}{\\exp$\_$overlap est une relation réflexive}
	\Axiom{map\_expOverlap\_thm\_symmetric}{true}{$\\\forall{}e1,e2 \qdot{} e1 \in{} exp\_overlap[\{e2\}]\\~     \limp{} e2 \in{} exp\_overlap[\{e1\}]$}{\\exp$\_$overlap est une relation symétrique}
	\Axiom{map\_expSubsets\_thm\_expSubsetsOverlaps}{true}{$\\\forall{}e1,e2 \qdot{} e1 \in{} exp\_subsets[\{e2\}]\\~     \limp{} e1 \in{} exp\_overlap[\{e2\}]$}{\\deux expressions qui sont en relation de sous-ensembles s'overlap}
	\Axiom{map\_expSubsets\_thm\_intersection}{false}{$\\\forall{}e1,e2,p \qdot{} p \in{} individuation[\{e1\}] \binter{} individuation[\{e2\}]\\~       \limp{} (\exists{}e \qdot{} p \in{} individuation[\{e\}] \land{}\\~                e \in{} exp\_subsets[\{e1\}] \land{}\\~                e \in{} exp\_subsets[\{e2\}])$}{\\toutes expressions partageant des particuliers peuvent être composées par une sous-expression}
	\Axiom{map\_expSubsets\_thm\_max\_decomposition}{false}{$\\\forall{}p \qdot{} p \in{} P\\~ \limp{} (\exists{}e \qdot{} \{p\} = individuation[\{e\}])$}{\\puisque toute expression est divisible, il existe une expression permettant l'individuation des particuliers de façon unitaire}
	\Axiom{map\_expSubsets\_thm\_decomposition}{true}{$\\\forall{}e,p \qdot{} e \in{} E \land{} \\~      \{p\} \subset{} individuation[\{e\}] \\~   \limp{} (\exists{}e' \qdot{} individuation[\{e'\}] \subset{} individuation[\{e\}] \land{}\\~             p \in{} individuation[\{e'\}])$}{\\toute expression permettant l'individuation de plus d'un particulier peut être décomposée}
	\Axiom{map\_expSubsets\_thm\_allTwoExpressionHasUnionExpression}{false}{$\\\forall{}e1, e2 \qdot{} e1 \in{} exp\_overlap[\{e2\}]\\~      \limp{} (\exists{}e3 \qdot{} e1 \in{} exp\_subsets[\{e3\}] \land{}\\~                e2 \in{} exp\_subsets[\{e3\}])$}{\\pour chaque pair d'expressions qui se chevauchent il existe une expression permettant d'individualiser l'union des particuliers de ces expressions }
	\Axiom{map\_equivalent\_def}{false}{$\\\forall{}e1,e2 \qdot{} \{e1,e2\} \subseteq{} E \land{}\\~        e2 \in{} exp\_equivalent[\{e1\}] \\~     \leqv{} \\~        e1 \in{} exp\_subsets[\{e2\}] \land{} e2 \in{} exp\_subsets[\{e1\}]$}{\\deux expressions sont équivalentes si leurs ensembles de particuliers sont sous-ensembles}
	\Axiom{map\_equivalent\_thm\_symmetric}{true}{$\\\forall{}e1,e2 \qdot{} \{e1,e2\} \subseteq{} E \land{}\\~        e2 \in{} exp\_equivalent[\{e1\}] \\~     \leqv{} \\~        e1 \in{} exp\_equivalent[\{e2\}]$}{\\exp$\_$equivalent est symétrique}
	\Axiom{map\_equivalent\_thm\_reflexive}{true}{$\forall{}e\qdot{}e \in{} E \limp{} e \in{} exp\_equivalent[\{e\}]$}{\\exp$\_$equivalent est réflexif}
	\Axiom{map\_equivalent\_thm\_transitive}{true}{$\\\forall{}e1,e2,e3 \qdot{} \{e1,e2,e3\} \subseteq{} E \land{}\\~           e1 \in{} exp\_equivalent[\{e2\}] \land{}\\~           e2 \in{} exp\_equivalent[\{e3\}]\\~        \limp{} e1 \in{} exp\_equivalent[\{e3\}]$}{\\exp$\_$equivalent est transitif}
	\Axiom{map\_neutralExpression\_thm\_subsetsAllExpression}{true}{$\\\forall{}e, empty\_e \qdot{} e \in{} E \land{}\\~             empty\_e \in{} E \land{}\\~             individuation[\{empty\_e\}] = \emptyset{}  \\~          \limp{} empty\_e \in{} exp\_subsets[\{e\}]$}{\\une expression n'individuant aucun particulier est un sous-ensemble de toutes les expressions }
	\Axiom{map\_individuation\_axm\_allClassesHasExpression}{false}{$\\\forall{}c \qdot{} c \in{} C \\~ \limp{} (\exists{} SE \qdot{} (\forall{}p \qdot{} c \in{} class\_of\_strict[\{p\}] \\~               \leqv{} p \in{} individuation[SE]))$}{\\pour chaque classe, il existe un ensemble d'expression qui permet l'individuation des particuliers stricts de cette classe}
	\Axiom{map\_keyCompatible\_axm\_reflexive}{false}{$\\\forall{}k \qdot{} k \in{} K\\~ \limp{} k \in{} key\_compatible[\{k\}]$}{\\une clé est nécessairement compatible avec elle-même}
	\Axiom{map\_keyCompatible\_axm\_symmetric}{false}{$\\\forall{}k1,k2 \qdot{} \{k1,k2\} \subseteq{} K \land{}\\~        k1 \in{} key\_compatible[\{k2\}]\\~     \limp{} k2 \in{} key\_compatible[\{k1\}]$}{\\la relation key$\_$compatibleest symétrique}
	\Axiom{map\_keyCompatible\_axm\_transitive}{false}{$\\\forall{}k1,k2,k3 \qdot{} \{k1,k2,k3\} \subseteq{} K \land{}\\~           k1 \in{} key\_compatible[\{k2\}] \land{}\\~           k2 \in{} key\_compatible[\{k3\}] \\~        \limp{} k1 \in{} key\_compatible[\{k3\}]$}{\\~relation key$\_$compatible est transitive}
	\Axiom{map\_expSubsets\_axm\_expressionsWithCompatibleKeyHasUnionExpressionWithCompatibleKey}{false}{$\\\forall{}e1, e2 \qdot{} e1 \in{} exp\_overlap[\{e2\}] \land{}\\~         e1 \notin{} exp\_subsets[\{e2\}] \land{}\\~         e2 \notin{} exp\_subsets[\{e1\}] \land{}\\~         exp\_key(e1) \in{} key\_compatible[\{exp\_key(e2)\}]\\~      \limp{} (\exists{}e3 \qdot{} e1 \in{} exp\_subsets[\{e3\}] \land{}\\~                e2 \in{} exp\_subsets[\{e3\}] \land{}\\~                exp\_key(e3) \in{} key\_compatible[\{exp\_key(e1)\}] \land{}\\~                exp\_key(e3) \in{} key\_compatible[\{exp\_key(e2)\}])$}{\\pour chaque pair d'expression qui se chevauche, sans qu'un soit le sous-ensemble de l'autre, si leurs clés sont compatibles il existe une expression permettant d'individualiser l'union des particuliers de ces expressions dont la clé est compatible avec celles des autres}
}
\END
\end{document}
